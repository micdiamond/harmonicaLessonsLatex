% BluesScaleRiffs without exercises
% Same as Lesson2 but without exercises
% Reorganised more basic for workshop
    \section{Blues Scale Riffs}
        \begin{itemize}
            \item Each riff we learn is based on the Blues Scale (we're learning the scale as we learn the riffs).
            \item We can use these riffs in solos
            % \item Remember: 3 blow is the same as 2 draw, but we prefer 2 draw
        \end{itemize}

        \section*{Bottom End - Blues Scale}
        The BOTTOM end of the blues scale is harder. We will learn 3 levels - Easy, Moderate, Hard.
            \subsection*{Easy - Bottom End}
            In the easy version, a couple of the notes from the blues scale are missing.
                
                \subsection*{Blues Scale - Bottom End - Going \Uparrow - Level: Easy}
                    \2 \3 \4
                    
                \subsection*{Blues Scale - Bottom End - Going \Downarrow Level: Easy}
                % {\textit{Level: Easy}}\\
                    \4\3\2 \\
                
        
        \newpage
            \subsection*{Moderate}
            In the moderate version, there is 1 note missing from the blues scale.

                \subsection*{Blues Scale - Bottom End - Going \Uparrow Level: Moderate}
                    \2\3\four\4
                    
                \subsection*{Blues Scale - Bottom End - Going \Downarrow Level: Moderate}
                    \4\four\3\2 \\
                
        
\newpage 

\section{Blues Riffs on The Blues Scale}
    The top end of the blues scale is easy. 
    \\ We will go up the scale, then down the scale. \Uparrow \Downarrow
        \section*{Top End - Blues Scale}
            \subsection*{Blues Scale - Top End \Uparrow}
                \4\5\six\\ \\
                
            \subsection*{Blues Scale - Top End \Downarrow}
                \six\5\4 \\
---         \subsection*{Blues Scale - Top End - 4 repeated repeated \Downarrow}
                \six\5\4\4 \\
                \six\5\4\4

            \subsection*{Be Bop A Lula - Straight timing (Notes on the beat - even notes - each note played with the same duration)}
                \six \5 \4 \5 \six \\
                
                
                
    \newpage
    \section{The Blue 3rd}
    When you play the 3 draw, try to bend it a tiny bit
    
    \\
    Like the song Spoon-Full \\
    That spoon, that spoon, that spoon full \\
    First try to play it without the 3 bend to see how it sounds \\
   
    \2 \3 \2 \3 \2 \3 \2
    
    \\
     Then try bending that 3 draw a tiny bit. Listen to the song "Spoon Full" if you want to hear how far to bend it. (But you will need a harmonica in the key of A to play along to it)
     
         \\
https://youtu.be/s0aIjyX7vwI
\\
     \\
    
    \2 \thdb \2 \thdb \2 \thdb \2
    
