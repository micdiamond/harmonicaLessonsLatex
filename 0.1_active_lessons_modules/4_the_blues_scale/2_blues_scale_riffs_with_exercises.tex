\newpage
    \section{Blues Riffs on The Blues Scale}
    The top end of the blues scale is easy. 
    \\ We will go up the scale, then down the scale. \Uparrow \Downarrow
        \section*{Top End - Blues Scale}
            \subsection*{Blues Scale - Top End \Uparrow}
                \4\5\six\\ \\
                
            \subsection*{Blues Scale - Top End \Downarrow}
                \six\5\4 \\
---         \subsection*{Blues Scale - Top End - 4 repeated repeated \Downarrow}
                \six\5\4\4 \\
                \six\5\4\4

            \subsection*{Be Bop A Lula - Straight timing (Notes on the beat - even notes - each note played with the same duration)}
                \six \5 \4 \5 \six \\
                Gene Vincent - Be Bop A Lula (Olympia , 1965): \\
                https://www.youtube.com/watch?v=-2aaIucjtGg
            \subsection*{Be Bop A Lula - swing timing (long and short notes}
                \six... \5.. \4. \5.. \six... \\
                (the dots are not exact, but indicate that the 6 is the longest and the 4 is the shortest)
    
\newpage    
        \section*{Bottom End - Blues Scale}
        The BOTTOM end of the blues scale is harder. We will learn 3 levels - Easy, Moderate, Hard.
            \subsection*{Easy - Bottom End}
            In the easy version, a couple of the notes from the blues scale are missing.
                
                \subsection*{Blues Scale - Bottom End \Downarrow Level: Easy}
                {\textit{Level: Easy}}\\
                    \4\3\2 \\
                
                \subsection*{Blues Scale - Bottom End \Uparrow Level: Easy}
                    \2 \3 \4
                    
              \subsection*{Blues riff - Straight timing}    
                   \4\3\2\3\4
                   
            \subsection*{Blues riff - Straight timing, \\but \textit{really} focusing on playing notes individually with a little gap between each note}    
                   \4\3\2\3\4   
                   
            \subsection*{Tounging the notes. (play it softly, but sharply)}    
                \4\3\2\3\4          \\
                ta ta ta ta ta ta     \\
                Using your tongue to begin the airflow sharply. Don't hit it with too much air though, because it will sound constricted as it tries to get through the reed. Also, you might fracture the reed if you hit it with too much air. 
                   
            \subsection*{Same, but SLIDING between the notes -  not stopping airflow}
                Need to push the harmonica across to the next note quickly, so it doesn't sound blurry. \\
                   \4\3\2\3\4   \\
                Should I be pushing the harmonica or moving my head? Either is fine, but I like to push the harmonica. \textit{(Next week, how to hold the harmonica properly to improve control}
                  
            \subsection*{Tonguing (play it softly, but sharply)}    
                \4\3\2\3\4          
                ta ta ta ta ta ta    
            
            \subsection*{Swung - Sliding between the notes}    
                \4...\3..\2.\3..\4 ...   
        
            \subsection*{2 draw}
                How's the two draw sounding?\\
                As the riffs get harder, you might be constricting your lips more when trying to play single notes. Any constriction will make the 2 draw sound like a duck. Constriction can occur at the lips, the front of the mouth, or at the back of the throat. For the best tone, you want no constriction (even when you bend notes) 

        
        
        \newpage
            \subsection*{Moderate Difficulty}
            In the moderate version, there is 1 note missing from the blues scale.
                \subsection*{Blues Scale - Bottom End \Downarrow Level: Moderate}
                    \4\four\3\2 \\
                
                \subsection*{Blues Scale - Bottom End \Uparrow Level: Moderate}
                    \2\3\four\4
        
        
        \newpage    
            \subsection*{Hard}
            In the hard version, there are no notes missing from the blues scale.
            This version includes the 4 draw bend - (\textit{Also known as the 'Tritone' or the 'Devils Interval'}).
                \subsection*{Blues Scale - Bottom End \Downarrow Level: Hard}
                    \4\e\four\3\2 \\
                
                \subsection*{Blues Scale - Bottom End \Uparrow Level: Hard}
                    \2\3\four\e\4
            
        
    % \newpage    
    %     \subsection{Piano man}
    %             \sixblow
    