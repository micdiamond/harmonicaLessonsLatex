% 3_Best_Blues_Riff_(required_bends)-Simplify_for_workshop

    \subsection{THE BEST blues riff in the blues}
        % Play this with one constant stream of air. 
        Play this riff using one constant stream of air. On the final note of each riff (the two draw)
        % \2 
        - you can hold this note, or you can cut it short and breathe out. Since this riff is all draw in breaths, breathe out all your air before you start. This trick of breathing out before you begin will be handy for most blues harmonica playing.\\
            \4\fdb\3\2  (breathe out)        \\
            \4\fdb\3\2  (breathe out)        \\
            \4\fdb\3\2   (breathe out)       \\
            \4\fdb\3\2  (breathe out)        \\
            
        If you're having trouble, Isolate the first three notes \\
                    \4\fdb\3\ \\
                    You can even go around and around on those three notes for practice.
                    Try to play these three notes over and over without taking a breath. \\
                         \4 \fdb \3 \4 \fdb \3 \\
                         \4 \fdb \3 \4 \fdb \3
            
\newpage
        \subsection{New blues riff}
        \2 \3 \4  (breathe out)\\
        \4\fdb\3\2  (breathe out)        \\
        \2 \3 \4   (breathe out)\\
        \4\fdb\3\2  (breathe out)        \\    
            
                \subsection{Simple blues riff}
        \\ A simple reminder that the 4 blow is also a note in the blues scale. \\
            \4\four\3\2         \\
            \4\four\3\2         \\
            \4\four\3\2         \\
            \4\four\3\2         \\    
            