\part*{Lesson 4}

\textbf{New symbols} \\            
\flat$ Flat (bent note) \\
\natural$ natural (unbent note) \\
\nearrow$ releasing a bend \\
\flattonatural - flat to natural = releasing a bend \\

    \ffd \\
    \ffdb \\
    \ffdbrB = \ffdbr

    \margincloud{These notations are equivalent}
    
    
    \4\5\4\5\textellipsis = \warble
    
\newpage

\section{Bending into the warble}
\fdb \4\5\4\5\4\5 \\

\warble

% MICHAEL TO CHANGE THIS TO START ON A 5 - del
\bendintowarble

\newpage
        \4\fdb\3\2   \\
        \2 \tdb \1 \1

        \subsection{Blues chords and melody}

        \2 \3 \4  \\
        \ottd \ottb \ottd (rest)\\    
        \4\fdb\3\2   \\
        \ottd \ottb \ottd (rest)\\    
        \2 \3 \4\\\
        \ottd \ottb \ottd (rest)\\    
        \4\fdb\3\2        \\    
    
        \ottd \ottb \ottd (rest)\\    
            
            
            
            
        \subsection{blues riff - tongued}
            \fdta\fdta\fdta\fdta  \\
            
\newpage
\section{The Blues scale}\\

\subsection{Going up - Main Octave}\\
\2\3\four\e\4\5\six\\

\subsection{Going down - Main Octave}\\
\six\5\4\e\four\3\2

\subsection{Going down (lower octave) - easy version}\\
\2\w\1\\


\subsection{Blues scale - most important notes (Going Up)}\\
\1\w \2 \\
\2\3\four\e\4\5\\

\subsection{Blues scale - most important notes (Going Down)}\\
\5\4\e\four\3\2 \\
\2\w\1\\

\newpage
\4 \fdb \4\\
entry into the note is very slow (long attack on the note) - a smooth transition from 0 airflow to the note sounding.\\

% \2\3\4 
\4\3\2


\newpage
Really slow. Slow enough for perfection. 

\subsection{Blues scale - most important notes (Going Up)}\\
\2\3\e\4\\\

\subsection{new riff 1}
\e\4 \4 \\ \4\e\3\2

\subsection{new riff 2}
\e\4 \5 \\ \4\e\3\2