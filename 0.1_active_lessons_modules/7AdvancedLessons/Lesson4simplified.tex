\part*{Lesson 4}
This week is about drilling the 2 and 4 draw bends and drilling the most important parts of the blues scale. \\



\textbf{Goals for this week:}\\
-Play the 1 and 2 draw bend clearly\\
-Practice riffs that use the 2 and 4 draw bends\\
-Play very slow. Slow enough for perfection. \\
-Play softly \\
-play with very little 'attack' at the start of the note\\

% \newpage
\section{Riffs to drill this week}
\textbf{Start each practice session by playing each riff 8 times.
}
\section{Riff 1 to drill this week}
        \4\fdb\3\2   \\

\section{Riff 2 to drill this week}
        \4\4\4\4\\
        \4\fdb\3\2   \\
\section{Riff 3 to drill this week}
        \2 \tdb \1 \1\\
        
        
\section{Riff 4 to drill this week}
        \2\2\2\2\\
        \2 \tdb \1 \1\\
        
\section{Riff 5 to drill this week}        
\1\odb\1\\
\1\odb\one

\section{Riff 6 to drill this week- a riff to practice playing quietly and clearly}        
Entry into the note is very slow (long attack on the note) - a smooth transition from 0 airflow to the note sounding.\\
\4 \fdb \4\\


\section{Riff 7 to drill this week- a riff to practice playing quietly and clearly}        
% \2\3\4 
\4\3\2

\section{Riff 7.1}
\4\3\2\3\2 \\
\textbf{Some different ways to play this:} \\
% \bi
- All notes connected (one stream of air) \\
- All notes individually\\
\textbf{- All notes tongued (Normal - full duration of the note held}\\
- All notes tongued (Stoccato - very short notes)
% \ei

\subsection{new riff 8}
\e\4 \4 \\ 
\4\e\3\2 \\

\subsection{new riff 8.1 - first note tongued}

\fdbta  \nearrow$ \4 \fdta \\
\fdbta  \nearrow$ \4 \fdta \\
\fdbta  \nearrow$ \4 \fdta \\
\4\nearrow$\e\3\2 \\

\subsection{new riff 8.2 - first note not tongued}
MINDBENDER - Don't focus on this one. \\
\fdb  \nearrow$ \4 \fdta \nearrow$\\
\fdb  \nearrow$ \4 \fdta \nearrow$\\
\fdb  \nearrow$ \4 \fdta \nearrow$\\
\4\nearrow$\e\3\2 \\

\subsection{new riff 8}
All of that is connected
\e\4 \5 \\ \4\e\3\2





    \newpage

\section{Bending into the warble}
\fdb \4\5\4\5\4\5 \\

\warble
\\
\bendintowarble



\newpage
\section{New symbols} \\            
% \flat$ Flat (bent note) \\
% \natural$ natural (unbent note) \\
\nearrow releasing a bend \\
% \flattonatural - flat to natural = releasing a bend \\

    \ffd \\
    \ffdb \\
    \ffdbrB = \ffdbr

    \margincloud{These notations are equivalent}
    
    
    \4\5\4\5\textellipsis = \warble
    
    
    

