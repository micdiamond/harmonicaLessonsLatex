
\section{Common beginner Mistakes}
\textbf{Reason 1: They play the wrong notes. }
There is a common misconception that “All notes sounds the harmonica sound good”. Without getting into the music theory behind why this only is sometimes correct, beginners often play notes that are not in the musical scale. The Diamonica customised harmonica can not play notes which lie outside of the blues scale - these notes have been blocked with stickers. After you learn these notes, you can open up the harmonica and take off the stickers.

\textbf{Reason 2: The can not bend the 3 draw bend down to “the blue third”.}
To play the blues scale correctly, you need to bend the 3 draw very precisely. This technique takes some time. On the Diamonica Training Harmonica, I have tuned the 3 draw reed so that it plays this note without needing to bend the note. This allows a beginner to play the blues scale right away. It also allows you to hear the pitch of the blue third, so that when you learn to bend notes, you already know how far to bend the note. 

\textbf{Reason 3: They can not play single notes.}
Learning the lip technique to play single notes requires practice. 
As a beginner, you will accidently be playing multiple notes at once (the note you are trying to play, plus the note beside the note you are trying to play.) Additionally, beginners don’t know what a single note should sound like. The notes either side of the 4 blow and the 6 blow are blocked, so when you play these notes, you will hear what a single notes should sound like. You are aiming to make all of your notes sound like this. 